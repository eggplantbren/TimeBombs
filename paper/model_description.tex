% Every Latex document starts with a documentclass command
\documentclass[a4paper, 11pt]{article}

% Load some packages
\usepackage{graphicx} % This allows you to put figures in
\usepackage{natbib}   % This allows for relatively pain-free reference lists
\usepackage[left=3cm,top=3cm,right=3cm]{geometry} % The way I like the margins
\usepackage{dsfont}
\usepackage{amsmath}

% This helps with figure placement
\renewcommand{\topfraction}{0.85}
\renewcommand{\textfraction}{0.1}
\parindent=0cm

% Set values so you can have a title
\title{}
\author{}
\date{\today}

% Document starts here
\begin{document}

% Actually put the title in
\maketitle

%\abstract{This is the abstract}

\section{Model description}

The conditional prior for the data (the photon counts $\{y_i, ..., y_n\}$)
given the parameters\footnote{Conventionally:
sampling distribution, likelihood.} $\boldsymbol{\theta}$ is Poisson,
where the parameters (including the time delay $\tau$)
determine the Poisson rate for each time bin:
\begin{eqnarray}
p(y_i | \boldsymbol{\theta}) \sim \textnormal{Poisson}
\left(
\lambda_i(\boldsymbol{\theta})
\right).
\end{eqnarray}



At infinite time resolution, the Poisson intensity is a sum of a constant
background level $b$ and $N$ asymmetric biexponential basis functions or
``spikes''. Without any time delay (i.e. if the source were not lensed),
the Poisson intensity would be:

\begin{eqnarray}
\mu_0(t) &=& b + \sum_{i=1}^N f(t; \boldsymbol{\theta}_i)
\end{eqnarray}
where $\boldsymbol{\theta}_i$ is the parameter vector for spike $i$.
With a doubly-imaged system, the non-constant part of the signal appears
again but delayed by $\tau$, the time delay of the system, and scaled by
a constant $C$, the magnification ratio of the two images:
\begin{eqnarray}
\mu(t) &=& b + \sum_{i=1}^N f(t; \boldsymbol{\theta}_i) +
C\sum_{i=1}^N f(t - \tau; \boldsymbol{\theta}_i)
\end{eqnarray}

When observed over a time interval from $t_i - h/2$ to $t_i + h/2$ (where
$h$ is the bin width and $t_i$ the time of the bin centre), the expected
number of photons (given the parameters) is:
\begin{eqnarray}
\lambda_i &=& \int_{t_i - h/2}^{t_i + h/2} \mu(t) \, dt.\label{eq:bin_integral}
\end{eqnarray}

\section{The Spike Shape}
A spike centered at time $t=t_c$ with amplitude $A$, characteristic timescale
(rise time) $L$ and asymmetry parameter $s$ has the following shape:
\begin{eqnarray}
f(t) &=& \left\{
\begin{array}{lr}
A\exp\left(\frac{t - t_c}{L}\right), & t < t_c\\
A\exp\left(\frac{-(t - t_c)}{sL}\right), & t \geq t_c
\end{array}
\right.
\end{eqnarray}
Therefore the parameter vector of a spike is
$\boldsymbol{\theta}_i = \{A_i, t_c^i, L_i, s_i\}$. One advantage of this
choice of spike shape is that definite integrals (which are needed to compute
the likelihood, because of Equation~\ref{eq:bin_integral} of $f(t)$ are
available analytically.

\bibliographystyle{plainnat}
\bibliography{references}{}


\end{document}

