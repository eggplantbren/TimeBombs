% Every Latex document starts with a documentclass command
\documentclass[a4paper, 11pt]{article}

% Load some packages
\usepackage{graphicx} % This allows you to put figures in
\usepackage{natbib}   % This allows for relatively pain-free reference lists
\usepackage[left=3cm,top=3cm,right=3cm]{geometry} % The way I like the margins
\usepackage{dsfont}
\usepackage{amsmath}

% This helps with figure placement
\renewcommand{\topfraction}{0.85}
\renewcommand{\textfraction}{0.1}
\parindent=0cm

% Set values so you can have a title
\title{}
\author{}
\date{\today}

% Document starts here
\begin{document}

% Actually put the title in
\maketitle

%\abstract{This is the abstract}

\section{Model description}

The conditional prior for the data given the parameters\footnote{Conventionally:
sampling distribution, likelihood.} is Poisson,
where the parameters (including the time delay $\tau$)
determine the Poisson rate for each time bin.
At infinite time resolution, the Poisson intensity is a sum of a constant
background level $b$ and $N$ asymmetric biexponential basis functions or
``spikes''. Without any time delay (i.e. if the source were not lensed),
the Poisson intensity would be:

\begin{eqnarray}
\mu_0(t) &=& b + \sum_{i=1}^N A_i f(t; \boldsymbol{\theta}_i)
\end{eqnarray}
where $\boldsymbol{\theta}_i$ is the parameter vector for spike $i$.
With a doubly-imaged system, the non-constant part of the signal appears
again but delayed by $\tau$, the time delay of the system, and scaled by
a constant $C$, the magnification ratio of the two images:
\begin{eqnarray}
\mu(t) &=& b + \sum_{i=1}^N A_i f(t; \boldsymbol{\theta}_i) +
C\sum_{i=1}^N A_i f(t - \tau; \boldsymbol{\theta}_i)
\end{eqnarray}

When observed over a time interval from $t_i - h/2$ to $t_i + h/2$ (where
$h$ is the bin width and $t_i$ the time of the bin centre), the expected
number of photons (given the parameters) is:
\begin{eqnarray}
\lambda_i &=& K_i(\boldsymbol{\beta})\int_{t_i - h/2}^{t_i + h/2} \mu(t) \, dt
\end{eqnarray}
The coefficient $K_i(\boldsymbol{\beta})$ is an ``extra noise'' factor, which
depends on parameters $\boldsymbol{\beta}$, designed to account for variability
that may exist but is not naturally explained by the spike model.

\bibliographystyle{plainnat}
\bibliography{references}{}


\end{document}

